
The number of spin revolutions per turn, spin tune $\nu_s$, depends on the particle's  equilibrium-level energy, expressed by the Lorentz factor:
\begin{align}\label{eq:spin_tune_vs_gamma}
\nu_s^B &= G\gamma, \tag{magnetic field}\\
\nu_s^E &= \frac{G+1}{\gamma} - G\gamma. \tag{electric field}
\end{align}

Not all beam particles in a bunch are characterized by the same Lorentz factor. A particle involved in betatron
motion will have a longer orbit, and as a direct consequence of the phase stability principle,
in an accelerating structure utilizing an RF cavity, its equilibrium energy level 
must increase.

Consider the reference particle. Its longitudinal dynamics is described
by the system of equations:
\begin{equation}
\begin{cases}
\ddt{}\D\varphi &= -\w_{RF}\eta\delta, \\
\ddt{}\delta &= \frac{q V_{RF}\w_{RF}}{2\pi h\beta^2E}(\sin\varphi - \sin\varphi_0).
\end{cases}
\end{equation}
In the equations above, $\D\varphi = \varphi - \varphi_0$ and
$\delta = (p-p_0)/{p_0}$ are the deviations of the particle's phase and
normalized momentum from those of the reference particle; all other symbols have their usual meanings.
%% $V_{RF}$, $\w_{RF}$ are, respectively,
%% the RF voltage and frequency; $\eta = \alpha_0 - \gamma^{-2}$ is the slip-factor,
%% where $\alpha_0$ is the momentum compaction factor defined by $\sfrac{\Delta L}{L} = \alpha_0\delta$,
%% $L$ being the orbit length; $h$ is the harmonic number; $E$ the total energy of the particle.

The solutions of this system form a family of ellipses in the $(\varphi, \delta)$-plane, all centered at the
point $(\varphi_0,\delta_0)$. However, if one considers a particle involved in betatron oscillations, and
uses a higher-order Taylor expansion of the momentum compaction factor
$\alpha = \alpha_0 + \alpha_1\delta$, the first equation of the system
transforms into:~\cite[p.~2579]{Senichev:IPAC13}
\begin{align*}
\ddt{\D\varphi} = -\w_{RF} \Bigg[\left(\frac{\Delta L}{L}\right)_\beta &+ (\alpha_0 + \gamma^{-2})\delta \Bigg.\\
&+ \Bigg.(\alpha_1 - \alpha_0\gamma^{-2} + \gamma^{-4})\delta^2\Bigg],
\end{align*}
where $(\frac{\Delta L}{L})_\beta = \frac{\pi}{2L}[\varepsilon_xQ_x + \varepsilon_yQ_y]$, is
the betatron motion-related orbit lengthening; $\varepsilon_x$ and $\varepsilon_y$ are
the horizontal and vertical beam emittances, and $Q_x$, $Q_y$ are the horizontal and vertical tunes.

The solutions of the transformed system are no longer centered at the same single point. Orbit lengthening
and momentum deviation cause an equilibrium-level momentum shift~\cite[p.~2581]{Senichev:IPAC13}
\begin{equation}\label{eq:EquLevMom_shift}
\Delta\delta_{eq} = \frac{\gamma_0^2}{\gamma_0^2\alpha_0 - 1}\left[\frac{\delta_m^2}{2}(\alpha_1 - \alpha_0\gamma^{-2} + \gamma_0^{-4}) + \left(\frac{\Delta L}{L}\right)_\beta\right],
\end{equation}
where $\delta_m$ is the amplitude of synchrotron oscillations.

We call the equilibrium energy level associated with the momentum shift~\eqref{eq:EquLevMom_shift},
the \emph{effective Lorentz factor}:
\begin{equation}\label{eq:EffectiveGamma}
\geff= \gamma_0 + \beta_0^2\gamma_0\cdot\Delta\delta_{eq},
\end{equation}
where $\gamma_0$, $\beta_0$ are the Lorentz factor and relative velocity factor of the reference particle.
